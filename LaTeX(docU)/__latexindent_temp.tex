\documentclass[conference]{IEEEtran}
\IEEEoverridecommandlockouts
\usepackage[spanish]{babel}
\usepackage{cite}
\usepackage{pgfopts}
\usepackage{amsmath,amssymb,amsfonts}
\usepackage{algorithmic}
\usepackage{graphicx}
\usepackage{textcomp}
\usepackage{xcolor}
\def\BibTeX{{\rm B\kern-.05em{\sc i\kern-.025em b}\kern-.08em
    T\kern-.1667em\lower.7ex\hbox{E}\kern-.125emX}}

\begin{document}

\title{Métodos de  Programación Orientada a Objetos.\\
    {\footnotesize Practica Cinco}}


\author{\IEEEauthorblockN{Ortega Ezquerra, Jesús Rodigo}
    \IEEEauthorblockA{\textit{Ingeniería Eléctrica y Electrónica.} \\
    \textit{Univesidad Nacional Autónoma de México}\\
    CDMX, México \\
    jeroorez@gmail.com\\
    https://orcid.org/0000-0002-8965-1189}
    \and
    \IEEEauthorblockN{Morales Medina, Antonio}
    \IEEEauthorblockA{\textit{Ingeniería Eléctrica y electrónica.} \\
    \textit{Univesidad Nacional Autónoma de México}\\
    CDMX, México \\
    moralesmedinaa201@gmail.com}
}\maketitle

\begin{abstract}
    En esta práctica se busca entender el uso del polimorfísmo en el lenguaje 
    de java para la clase de Modelos de Programación Orientada a Objetos.
\end{abstract}
    
\begin{IEEEkeywords}
    POO, java 8, polimorfismo, instenceof, abstract, extends, super
\end{IEEEkeywords}



\section{Introducción.}

    Para el previo de esta práctica se debían preparar tres clases:
    
    \begin{enumerate}
        \item Poligono(superclase)
        \item Triangulo
        \item Cuadrilatero
    \end{enumerate}
    
    La clase \texttt{App} se escribió en clase cómo \texttt{MPOOP7}  para efectuar las actividades uno y dos 
    \subsection{Objetivos.}

    \subsection{}
\section{Clases base.}

    \subsection{Tipos de clase.}

    \begin{enumerate}
        
        \item \textit{Clase Base}\\
            También llamada clase padre o superclase, es aquella que va a heredar a las clases hijas que derivan a partir de la clase padre.
        
        \item \textit{Clase Derivada}\\
            Clase que reinstancía a la clase padre de la que deriva para ser una de sus subclases, cabe mencionar que hereda los métodos y variables de la clase padre.
            Se le conoce también como clase hija o subclase.

    \end{enumerate}

\section{Método instanceof.}

    \subsection{\texttt{Instanceof}}
        
        \begin{verbatim}
            "The instanceof keyword compares an object to a class or interface type. It also looks at
            subclasses and subinterfaces. x instanceof Object returns true unless x is null"
        \end{verbatim}

\section{Clase abstracta.}

\section{Interfaz.}

\section{Atributos de interfaz.}

    
\section{Concluciones.}
    \subsection{}

\end{document}